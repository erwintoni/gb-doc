\Chapter{Introduction}

\section{Background}

A long time ago, businesses used to produce their own electric power. And due to engineering breakthrough in electric generator and transmission method, it became easier to produce and transmit electricity, to supply businesses that once produced their own electricity. As more businesses started buying electric power, making utility expanded and electricity cheaper. 

And today, just like the utilities, instead of buying servers to run your websites or applications, you rent servers or server spaces from cloud computing providers. Just like renting an apartment, even you are in the same building with other people, you still have your own space. As more people rent and buy computing power, making clould computing expanded and    

Continue with the stories till cloud computing points.


The following describe the clould computing service providers used in this project. Heroku, MongoLab, and GitHub.

\section{Facebook}

\section{Heroku}
Cloud computing is a model which makes use of computer hardware and software that are accessed through the Internet as services. There are several choices of cloud computing services available, but for this project we choose the one provided by Heroku, the cloud computing partner of Facebook. \cite{Heroku}.

The reasons are that Heroku lets you use and publish an application that people can use right away with no cost and obligation, and you can take advantage of the same scalable technologies that Facebook applications are built on, and attain a similar level of reliability, performance and security. 

\section{MongoDB}
There are many different types of cloud-based datastore services to choose from. For this project we will use MongoDB, as it works well Node.js and Heroku. MongoDB is a scalable, high-performance, open source, NoSQL document-based database. MongoDB features include document-oriented storage, indexes, replication, high availability, auto-sharding, and querying.

\section{MongoLab}

\section{Git}
Git is a distributed version control system.  This project uses git with GitHub, a cloud-based provider of remote git repository storage.  Heroku uses git as a means to deploy web applications to its servers. Git allows easy creation of testing, staging, and production versions of the application. 

\section{Bootstrap}


\section{Jquery}

\section{Node.js}
Cloud-based services support apps written in several different programming languages, such as Java, Python, PHP, Javascript, Ruby and many more. For this project we would use Javascript running in a Node.js context. Node.js is a platform built on Chrome's JavaScript runtime for easily building fast, scalable network applications. Node.js uses an event-driven, non-blocking I/O model that makes it lightweight and efficient, perfect for data-intensive real-time applications that run across distributed devices.

\section{Purpose}
To explore the new technologies, to create cross-platform reward application that individual can use.

\section{Project Scope}
Project does not include database sharding features to allow greater degree of scalability. 

The GradeBadge application provides the following functionalities:
\begin{itemize}
\item Create group
\item Create Badge 
\item Add Member
\item Issue Badges to Members
\item View Badge Earned
\item Share Badge to Social Networking 
\end{itemize}

\section{Related Work}
Explain manoj work and explain the differences, Google App Engine VS Heroku, Google Data Store VS MongoLab, Java VS Node.js, Jquery Mobile VS Bootstrap. \cite{Manoj}

\section{Project Limitations}
Users must have Facebook account,  logged in to facebook and authorized access to basic information (name, profile picture and friend list) .For best experience must use modern browser in either PC or tablet or smart phone.    

\section{Definitions, Acronyms, and Abbreviations}

The definitions, acronyms, and abbreviations used in the document are described in this section.

\begin{itemize}
\item GradeBadge: The name of this project
\item API: Application Programming Interface is a set of routines that an application uses to request and carry out low-level services performed by a computer's operating system; also, a set of calling conventions in programming that defines how a service is invoked through the application \cite{API}.
\item Cloud computing: Cloud computing is the use of computing resources (hardware and software) that are delivered as a service over a network (typically the Internet) \cite{cloudcomputing}.
\item JQuery: A javascript library provided by JQuery for building web based applications \cite{JQuery}.
\item UI: User Interface
\item CSUSB: California State University, San Bernardino.
\item HTML: HyperText Markup Language is the authoring language used to create documents on the World Wide Web \cite{w3}.
\item HTTPS: Hyper Text Transfer Protocol Secure is a secure network protocol used to encrypt data transferred  between server and client \cite{https}.
\item MVC\label{def:mvc}: Model-View-Controller is an architectural pattern used in software engineering to isolate business logic from user interface considerations \cite{mvc}.
\item UML: The Unified Modeling Language is the industry-standard language for specifying, visualizing, constructing, and documenting the artifacts of software systems \cite{uml}.
\item Microsoft Azure: Cloud Computing platform provided by Microsoft \cite{MicrosoftAzure}.
\item Amazon Web Services: Cloud Computing platform provided by Amazon \cite{AWS}.
\item Heroku:Cloud Application platform provided by Heroku \cite{Heroku}.
\item Android : Mobile Operating System provided by Google \cite{Android}.
\item IOS : Mobile Operating System provided by Apple \cite{IOS}.
\item PhoneGap: Open Source Framework for creating Mobile Apps \cite{PhoneGap}.
\item NoSQL: Uses key-value pairs for storing data unlike traditional Relational Database Management \cite{NoSql}.
\item JSON : Javascript Object Notation built using key and value pairs \cite{json}.
\item Ajax:  Asynchronous JavaScript and XML/JSON format for communicating from client to the server \cite{Ajax}.
\item OOP : Object Oriented Programming concept with objects representing real world entities. Methods expose state of the object \cite{OOP}.
\end{itemize}


