\Chapter{Database Design}

Explain Document-oriented design 

\section{MongoDB}
 MongoDB is a scalable, high-performance, open source, NoSQL document-based database. MongoDB features include document-oriented storage, indexes, replication, high availability, auto-sharding, and querying.

\section{MongoLab}
MongoLab is the cloud computing provider for MongoDB database, easily integrated with Heroku.

\section{Documents}
Data in MongoDB is stored in documents.

\section{Collections}
Documents in MongoDB are organized in colletion

\section{Data Model}
In this project, there are three main collections, User, Group and Badge Colletion. And there are four linkings colletions to represent the many-to-many relationship between the main collections.

Collections diagram

\begin{itemize}
\item user : this collection contains the user information
\item group : this collection contains the group information  
\item badge : this collection contains the badge information
\item group{\_}admin{\_}links : this linking collection contains group{\_}id and user{\_}id, which represent the admin of a group
\item group{\_}member{\_}links : this linking collection contains group{\_}id and user{\_}id, which represent the member of a group
\item badge{\_}user{\_}links : this linking colletion contains badge{\_}id and user{\_}id, which reperesnt badges that user earned
\item group{\_}badge{\_}links : think linking collection contains group{\_}id and badge{\_}id, which represent badges that belong to a group
\end{itemize}




